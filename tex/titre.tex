\newgeometry{margin=2em}

\begin{center}
	\begin{minipage}[c]{.5\linewidth}
		\raggedright\includegraphics[height=7em]{logo_amu_excellence}
	\end{minipage}
	\hfill
	\begin{minipage}[c]{.5\linewidth}
		%% Logo d'une éventuelle cotutelle
		%% --------------------------------------------------------------------
		%\raggedleft\includegraphics[height=7em]{example-image-b}
		%% --------------------------------------------------------------------
	\end{minipage}\hfill
\end{center}

\vspace{10pt}

\begin{center}
	\begin{minipage}[c]{.66\linewidth}
		\vspace*{1pt}%
		\dhorline{\textwidth}{4pt}
	\end{minipage}
	\hfill
	\begin{minipage}[c]{.33\linewidth}
		\raggedleft\color{blueamu}\small
		\titm{\titb{NNT/}\titl{NL}:
		% Renseigner votre numéro national de thèse (NNT)
		% et le numéro local (NL).
		% Ils sont indiqués sur la page d'informations administratives de votre
		% espace de dépôt dans le guichet de dépôt légal des thèses AMU lorque
		% votre date de soutenance est renseignée par votre service de scolarité
		% → https://depot-theses.univ-amu.fr/
		%% --------------------------------------------------------------------
		\titb{2020AIXM0001/}\titl{001ED000}}
		%% --------------------------------------------------------------------
	\end{minipage}
\end{center}

\vspace{10pt}

\begin{flushleft}
	\titb{\HUGE\textcolor{cyanamu}{THÈSE DE DOCTORAT}}
    \vspace{\stretch{1}}
    
	\titl{\Large Soutenue à Aix-Marseille Université}
	%% Préciser si cotutelle :
	%% -------------------------------------------------------------------------
 	%\\\titl{\Large dans le cadre d'une cotutelle avec}
	%% -------------------------------------------------------------------------
	
	%% Insérer la date *de soutenance* :
	%% -------------------------------------------------------------------------
	\titl{\Large le <date> par}
	%% -------------------------------------------------------------------------
\end{flushleft}

\vspace{\stretch{4}}

\begin{center}
	%% -------------------------------------------------------------------------
	\titsb{\Huge <Prénom NOM>}
	%% -------------------------------------------------------------------------
	
    \vspace{\stretch{4}}
    
    %% -------------------------------------------------------------------------
	\titm{
		{\LARGE <Titre de la thèse>}
		\vspace{\stretch{1}}
		
		{\Large <Sous-titre de la thèse>}
	}
	%% -------------------------------------------------------------------------
\end{center}

\vspace{\stretch{8}}

\begin{varwidth}[t]{.48\linewidth}
		\titb{Discipline}
		
		\titl{Renseigner la discipline du doctorat}
	
	\vspace{5pt}
	
		\titb{Spécialité}
		
		\titl{Renseigner la spécialité du doctorat}
		
	\vspace{15pt}
		
		\titb{École doctorale}
		
		\titl{Renseigner l'école doctorale (ED 123)}
		
	\vspace{5pt}
		
		\titb{Laboratoire}
		
		\titl{Renseigner le laboratoire (UMR 1234)}
		
%	\vspace{5pt}
%	
%		\titb{Partenaires de recherche}
%		
%		\titl{Partenaire 1}
%		
%		\titl{Partenaire 2}
\end{varwidth}
\hfill
\begin{minipage}[t]{.04\linewidth}
	\centering
	%% Ajuster les longueurs à la main.
	%% Le \vspace{...} sert à positionner le début de la ligne pointillée
	%% (une longuer négative permet de le « remonter » un peu).
	%% Le \dvertline{4pt}{...} fixe la longueur de la ligne pointillée (la
	%% longueur doit être négative).
	%-----------------------------------------------------------------------
	\vspace*{-30pt}
	\dvertline{4pt}{-200pt}
	%-----------------------------------------------------------------------
\end{minipage}
\hfill
\begin{minipage}[t]{.48\linewidth}
	\titb{Composition du jury}
	
	\newcommand{\jury}[3]{%
	\vspace{6pt}%
	\titl{#1}\hfill
	\titel{#3}\par
	\vspace{1pt}%
	\titel{#2}
	}
	
	\jury{Prénom NOM}{Affiliation}{Rapporteur·e}
	
	\jury{Prénom NOM}{Affiliation}{Rapporteur·e}
	
	\jury{Prénom NOM}{Affiliation}{Président·e du jury}
	
	\jury{Prénom NOM}{Affiliation}{Examinateur·ice}
	
	\jury{Prénom NOM}{Affiliation}{Examinateur·ice}
	
	\jury{Prénom NOM}{Affiliation}{Directeur·ice de thèse}
	
	\jury{Prénom NOM}{Affiliation}{Directeur·ice de thèse}
\end{minipage}

\vspace{\stretch{2}}

%% Logos des partenaires : remplacer example-image-x par le chemin vers les
%% logos des institutions partenaires.
%\vspace{\stretch{2}}
%\begin{center}
%	\begin{minipage}[c]{.25\linewidth}
%		\centering\includegraphics[height=5em]{example-image-a} 
%	\end{minipage}
%	\hfill
%	\begin{minipage}[c]{.25\linewidth}
%		\centering\includegraphics[height=5em]{example-image-b}
%	\end{minipage}
%	\hfill
%	\begin{minipage}[c]{.25\linewidth}
%		\centering\includegraphics[height=5em]{example-image-c} 
%	\end{minipage}
%\end{center}

\restoregeometry