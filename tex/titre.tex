\newgeometry{margin=2em}

\begin{center}
	\begin{minipage}[c]{.5\linewidth}
		\raggedright\includegraphics[height=7em]{logo_amu_excellence}
	\end{minipage}\hfill
	\begin{minipage}[c]{.5\linewidth}
		%\raggedleft\includegraphics[height=7em]{example-image-b} %% logo cotutelle
	\end{minipage}\hfill
\end{center}

\vspace{1em}

\begin{center}
	\begin{minipage}[c]{.63\linewidth}
		\dhorline{\textwidth}{4pt}
	\end{minipage}\hfill
	\begin{minipage}[c]{.35\linewidth}
		\raggedleft\titl{NNT/NL : 2020AIXM0001/001ED000}
		% Renseigner votre numéro national de thèse (NNT)
		% et le numéro local (NL).
		% Ils sont indiqués sur la page d'informations administratives de votre
		% espace de dépôt dans le guichet de dépôt légal des thèses AMU lorque
		% votre date de soutenance est renseignée par votre service de scolarité
		% → https://depot-theses.univ-amu.fr/
	\end{minipage}\hfill
\end{center}

%\vspace{1em}

\doublespacing
\begin{flushleft}
    \titb{\HUGE\textcolor{cyanamu}{THÈSE DE DOCTORAT}}\\
	\titl{\Large Soutenue à Aix-Marseille Université}\\
	%\titl{\Large dans le cadre d'une cotutelle avec }\\
	\titl{\Large le 10 janvier 2020 par}\\
\end{flushleft}
\vspace{2em}
\begin{center}
	\titsb{\Huge Prénom NOM}\\
    \vspace{1em}
	\titm{\LARGE Titre de la thèse:\\ sous-titre de la thèse}\\
\end{center}
\singlespacing

\vspace{4em}

\begin{center}
	\begin{minipage}[t]{.45\linewidth}
    	    \vspace{.5em}
        	\titb{Discipline}
        	
        	\titl{renseigner la discipline du doctorat}
        	
    	    \vspace{1em}
        	\titb{Spécialité}
        	
        	\titl{renseigner la spécialité du doctorat}
        	
    	    \vspace{2em}
        	\titb{École doctorale}
        	
        	\titl{renseigner l'école doctorale}
        	
    	    \vspace{1em}
        	\titb{Laboratoire/Partenaires de recherche}
        	
        	\titl{renseigner les partenaires institutionnels
        	
        	et les partenaires privés
        	
        	un partenaire par ligne
        	}
	\end{minipage}\hfill
	\begin{minipage}[t]{.03\linewidth}
		% Remplacer -16em par une longueur (négative) qui va bien.
	    \vspace*{-1.5em}\dvertline{4pt}{-16em}
	\end{minipage}\hfill
	\begin{minipage}[t]{.52\linewidth}
	    \vspace{.5em}
    	\titb{Composition du jury}
		
		\newcommand{\jury}[3]{%
			\vspace{1em}%
			\titl{#1}\hfill
			\titel{#3}\\
			\titel{#2}\par
		}
	    
	    \jury{Prénom NOM}{Affiliation}{Rapporteur·e}
    	\jury{Prénom NOM}{Affiliation}{Rapporteur·e}
    	\jury{Prénom NOM}{Affiliation}{Président·e du jury}
    	\jury{Prénom NOM}{Affiliation}{Examinateur·ice}
    	\jury{Prénom NOM}{Affiliation}{Examinateur·ice}
    	\jury{Prénom NOM}{Affiliation}{Directeur·ice de thèse}
    	\jury{Prénom NOM}{Affiliation}{Directeur·ice de thèse}
	\end{minipage}\hfill
\end{center}       

\vspace{2em}

\begin{center} %% logos partenaires
	\begin{minipage}[c]{.25\linewidth}
		\centering\includegraphics[height=5em]{example-image-a} 
	\end{minipage}\hfill
	\begin{minipage}[c]{.25\linewidth}
		\centering\includegraphics[height=5em]{example-image-b}
	\end{minipage}\hfill
	\begin{minipage}[c]{.25\linewidth}
		\centering\includegraphics[height=5em]{example-image-c} 
	\end{minipage}\hfill
\end{center}

\restoregeometry