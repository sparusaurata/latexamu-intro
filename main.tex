\documentclass[paper=a4,fontsize=12pt,DIV=calc]{scrreprt}
\usepackage[utf8]{inputenc}
\usepackage[T1]{fontenc}
\usepackage{lmodern}
\usepackage[english,french]{babel}
\usepackage[babel]{csquotes}
\MakeAutoQuote{«}{»}	

\usepackage[a4paper, pass]{geometry}
\usepackage{graphicx}
\usepackage{xcolor}
\usepackage{multicol}
\usepackage{setspace}
\usepackage[colorlinks=true, urlcolor=blue]{hyperref}

\usepackage{tikz}
\usetikzlibrary{decorations.markings}
\definecolor{blueamu}{RGB}{0, 101, 189}
\definecolor{cyanamu}{RGB}{61, 183, 228}
\newcommand{\dhorline}[3][0]{%
    \tikz[baseline=-2pt]{\path[decoration={markings, 
      mark=between positions 0 and 1 step 2*#3
      with {\node[color=blueamu, fill, circle, minimum width=#3, inner sep=0pt, anchor=south west] {};}},postaction={decorate}]  (0,#1) -- ++(#2,0);}}
\newcommand{\dvertline}[3][0]{%
    \tikz[baseline=2em]{\path[decoration={markings,
      mark=between positions 0 and 1 step 2*#2
      with {\node[color=black!50, fill, circle, minimum width=#2, inner sep=0pt, anchor=south west] {};}},postaction={decorate}] (0, #1) -- ++(0,#3);}}  

\graphicspath{{fig/}{logotypes/}}

\newcommand\titel[1]{{\usefont{T1}{tit}{el}{n} #1 }}
\newcommand\titl[1]{{\usefont{T1}{tit}{l}{n} #1 }}
\newcommand\titm[1]{{\usefont{T1}{tit}{m}{n} #1 }}
\newcommand\titsb[1]{{\usefont{T1}{tit}{sb}{n} #1 }}
\newcommand\titb[1]{{\usefont{T1}{tit}{b}{n} #1 }}

\usepackage{lipsum}


\makeatletter\newcommand\HUGE{\@setfontsize\Huge{28}{0}}\makeatother

\begin{document}
    \newgeometry{margin=2em}
    \newgeometry{margin=2em}

\begin{center}
	\begin{minipage}[c]{.5\linewidth}
		\raggedright\includegraphics[height=7em]{logo_amu_excellence}
	\end{minipage}
	\hfill
	\begin{minipage}[c]{.5\linewidth}
		%% Logo d'une éventuelle cotutelle
		%% --------------------------------------------------------------------
		%\raggedleft\includegraphics[height=7em]{example-image-b}
		%% --------------------------------------------------------------------
	\end{minipage}\hfill
\end{center}

\vspace{10pt}

\begin{center}
	\begin{minipage}[c]{.66\linewidth}
		\vspace*{1pt}%
		\dhorline{\textwidth}{4pt}
	\end{minipage}
	\hfill
	\begin{minipage}[c]{.33\linewidth}
		\raggedleft\color{blueamu}\small
		\titm{\titb{NNT/}\titl{NL}:
		% Renseigner votre numéro national de thèse (NNT)
		% et le numéro local (NL).
		% Ils sont indiqués sur la page d'informations administratives de votre
		% espace de dépôt dans le guichet de dépôt légal des thèses AMU lorque
		% votre date de soutenance est renseignée par votre service de scolarité
		% → https://depot-theses.univ-amu.fr/
		%% --------------------------------------------------------------------
		\titb{2020AIXM0001/}\titl{001ED000}}
		%% --------------------------------------------------------------------
	\end{minipage}
\end{center}

\vspace{10pt}

\begin{flushleft}
	\titb{\HUGE\textcolor{cyanamu}{THÈSE DE DOCTORAT}}
    \vspace{\stretch{1}}
    
	\titl{\Large Soutenue à Aix-Marseille Université}
	%% Préciser si cotutelle :
	%% -------------------------------------------------------------------------
 	%\\\titl{\Large dans le cadre d'une cotutelle avec}
	%% -------------------------------------------------------------------------
	
	%% Insérer la date *de soutenance* :
	%% -------------------------------------------------------------------------
	\titl{\Large le <date> par}
	%% -------------------------------------------------------------------------
\end{flushleft}

\vspace{\stretch{4}}

\begin{center}
	%% -------------------------------------------------------------------------
	\titsb{\Huge <Prénom NOM>}
	%% -------------------------------------------------------------------------
	
    \vspace{\stretch{4}}
    
    %% -------------------------------------------------------------------------
	\titm{
		{\LARGE <Titre de la thèse>}
		\vspace{\stretch{1}}
		
		{\Large <Sous-titre de la thèse>}
	}
	%% -------------------------------------------------------------------------
\end{center}

\vspace{\stretch{8}}

\begin{varwidth}[t]{.48\linewidth}
		\titb{Discipline}
		
		\titl{Renseigner la discipline du doctorat}
	
	\vspace{5pt}
	
		\titb{Spécialité}
		
		\titl{Renseigner la spécialité du doctorat}
		
	\vspace{15pt}
		
		\titb{École doctorale}
		
		\titl{Renseigner l'école doctorale (ED 123)}
		
	\vspace{5pt}
		
		\titb{Laboratoire}
		
		\titl{Renseigner le laboratoire (UMR 1234)}
		
%	\vspace{5pt}
%	
%		\titb{Partenaires de recherche}
%		
%		\titl{Partenaire 1}
%		
%		\titl{Partenaire 2}
\end{varwidth}
\hfill
\begin{minipage}[t]{.04\linewidth}
	\centering
	%% Ajuster les longueurs à la main.
	%% Le \vspace{...} sert à positionner le début de la ligne pointillée
	%% (une longuer négative permet de le « remonter » un peu).
	%% Le \dvertline{4pt}{...} fixe la longueur de la ligne pointillée (la
	%% longueur doit être négative).
	%-----------------------------------------------------------------------
	\vspace*{-30pt}
	\dvertline{4pt}{-200pt}
	%-----------------------------------------------------------------------
\end{minipage}
\hfill
\begin{minipage}[t]{.48\linewidth}
	\titb{Composition du jury}
	
	\newcommand{\jury}[3]{%
	\vspace{6pt}%
	\titl{#1}\hfill
	\titel{#3}\par
	\vspace{1pt}%
	\titel{#2}
	}
	
	\jury{Prénom NOM}{Affiliation}{Rapporteur·e}
	
	\jury{Prénom NOM}{Affiliation}{Rapporteur·e}
	
	\jury{Prénom NOM}{Affiliation}{Président·e du jury}
	
	\jury{Prénom NOM}{Affiliation}{Examinateur·ice}
	
	\jury{Prénom NOM}{Affiliation}{Examinateur·ice}
	
	\jury{Prénom NOM}{Affiliation}{Directeur·ice de thèse}
	
	\jury{Prénom NOM}{Affiliation}{Directeur·ice de thèse}
\end{minipage}

\vspace{\stretch{2}}

%% Logos des partenaires : remplacer example-image-x par le chemin vers les
%% logos des institutions partenaires.
%\vspace{\stretch{2}}
%\begin{center}
%	\begin{minipage}[c]{.25\linewidth}
%		\centering\includegraphics[height=5em]{example-image-a} 
%	\end{minipage}
%	\hfill
%	\begin{minipage}[c]{.25\linewidth}
%		\centering\includegraphics[height=5em]{example-image-b}
%	\end{minipage}
%	\hfill
%	\begin{minipage}[c]{.25\linewidth}
%		\centering\includegraphics[height=5em]{example-image-c} 
%	\end{minipage}
%\end{center}

\restoregeometry
    \restoregeometry
    \newgeometry{bottom=10em}

    \iffalse
% Déclaration sur l'honneur pour une thèse en français (remplacer \iffalse par 
% \iftrue pour une thèse en anglais).
\begin{french}
	\chapter*{Déclaration sur l'honneur}
	
	Je soussigné,
	% Insérer les prénom et nom de l'auteur⋅e de la thèse.
	% --------------------------------------------------------------------------
	<Prénom Nom>,
	% --------------------------------------------------------------------------
	déclare par la présente que le travail présenté dans ce manuscrit est mon 
	propre travail, réalisé sous la direction scientifique de
	% Insérer les prénoms et noms du directeur de thèse et des éventuels
	% co-directeurs.
	% --------------------------------------------------------------------------
	<Prénom Nom>,
	% --------------------------------------------------------------------------
	dans le respect des principes d’honnêteté, d'intégrité et de responsabilité 
	inhérents à la mission de recherche. Les travaux de recherche et la 
	rédaction de ce manuscrit ont été réalisés dans le respect à la fois de la 
	charte nationale de déontologie des métiers de la recherche et de la charte 
	d’Aix-Marseille Université relative à la lutte contre le plagiat.
	
	Ce travail n'a pas été précédemment soumis en France ou à l'étranger dans 
	une version identique ou similaire à un organisme examinateur.
	
	% Préciser la ville (et la date si besoin).
	% --------------------------------------------------------------------------
	Fait à <ville> le \today.
	% --------------------------------------------------------------------------
	
	\begin{flushright}
	% Signature : remplacer example-image-a par le chemin vers une image
	% contenant votre signature.
	% --------------------------------------------------------------------------
	\tikz{\node[rectangle, draw, minimum width=4cm, minimum height=2cm] (r) at 
	(0,0) {signature};}
	% --------------------------------------------------------------------------
	\end{flushright}
\end{french}
\fi

\iftrue
% Affidavit of Honour for a thesis in English (invert the \iftrue and \iffalse 
% for a thesis in French).
	\chapter*{Affidavit}
	
	I, undersigned,
	% --------------------------------------------------------------------------
	<First Name Surname>,
	% --------------------------------------------------------------------------
	hereby declare that the work presented in this manuscript is my own work, 
	carried out under the scientific direction of
	% --------------------------------------------------------------------------
	<First Name Surname>,
	% --------------------------------------------------------------------------
	in accordance with the principles of honesty, integrity and responsibility 
	inherent to the research mission. The research work and the writing of this 
	manuscript have been carried out in compliance with both the french 
	national charter for Research Integrity and the Aix-Marseille University 
	charter on the fight against plagiarism.
	
	This work has not been submitted previously either in this country or in 
	aother country in the same or in a similar version to any other examination 
	body.
	
	% --------------------------------------------------------------------------
	<Place>, \today.
	% --------------------------------------------------------------------------
	
	\begin{flushright}
	% --------------------------------------------------------------------------
	\tikz{\node[rectangle, draw, minimum width=4cm, minimum height=2cm] (r) at 
	(0,0) {signature};}
	% --------------------------------------------------------------------------
	\end{flushright}
\fi

%~\vfill
%\begin{center}
%	\begin{minipage}[c]{0.25\linewidth}
%		\includegraphics[height=35px]{by-nc-nd-eu}
%	\end{minipage}\hfill
%\end{center}
%
%Cette \oe{}uvre est mise à disposition selon les termes de la \href{https://creativecommons.org/licenses/by-nc-nd/4.0/deed.fr}{Licence Creative Commons Attribution - Pas d’Utilisation Commerciale - Pas de Modification 4.0 International}. % consultez les conditions de la licence cc by-nc-nd, vous pouvez appliquer une licence moins restrictive, cc by-nc-sa par exemple


	\chapter*{Résumé}					%% résumé
    \input{resume}
\begingroup % permet de garder
\let\clearpage\relax % la LOF et la LOT sur la même
~\vfill
	\chapter*{Abstract}					%% résumé
    \input{abstract}
\endgroup % page

\end{document}